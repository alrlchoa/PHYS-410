
% Default to the notebook output style

    


% Inherit from the specified cell style.




    
\documentclass[11pt]{article}

    
    
    \usepackage[T1]{fontenc}
    % Nicer default font (+ math font) than Computer Modern for most use cases
    \usepackage{mathpazo}

    % Basic figure setup, for now with no caption control since it's done
    % automatically by Pandoc (which extracts ![](path) syntax from Markdown).
    \usepackage{graphicx}
    % We will generate all images so they have a width \maxwidth. This means
    % that they will get their normal width if they fit onto the page, but
    % are scaled down if they would overflow the margins.
    \makeatletter
    \def\maxwidth{\ifdim\Gin@nat@width>\linewidth\linewidth
    \else\Gin@nat@width\fi}
    \makeatother
    \let\Oldincludegraphics\includegraphics
    % Set max figure width to be 80% of text width, for now hardcoded.
    \renewcommand{\includegraphics}[1]{\Oldincludegraphics[width=.8\maxwidth]{#1}}
    % Ensure that by default, figures have no caption (until we provide a
    % proper Figure object with a Caption API and a way to capture that
    % in the conversion process - todo).
    \usepackage{caption}
    \DeclareCaptionLabelFormat{nolabel}{}
    \captionsetup{labelformat=nolabel}

    \usepackage{adjustbox} % Used to constrain images to a maximum size 
    \usepackage{xcolor} % Allow colors to be defined
    \usepackage{enumerate} % Needed for markdown enumerations to work
    \usepackage{geometry} % Used to adjust the document margins
    \usepackage{amsmath} % Equations
    \usepackage{amssymb} % Equations
    \usepackage{textcomp} % defines textquotesingle
    % Hack from http://tex.stackexchange.com/a/47451/13684:
    \AtBeginDocument{%
        \def\PYZsq{\textquotesingle}% Upright quotes in Pygmentized code
    }
    \usepackage{upquote} % Upright quotes for verbatim code
    \usepackage{eurosym} % defines \euro
    \usepackage[mathletters]{ucs} % Extended unicode (utf-8) support
    \usepackage[utf8x]{inputenc} % Allow utf-8 characters in the tex document
    \usepackage{fancyvrb} % verbatim replacement that allows latex
    \usepackage{grffile} % extends the file name processing of package graphics 
                         % to support a larger range 
    % The hyperref package gives us a pdf with properly built
    % internal navigation ('pdf bookmarks' for the table of contents,
    % internal cross-reference links, web links for URLs, etc.)
    \usepackage{hyperref}
    \usepackage{longtable} % longtable support required by pandoc >1.10
    \usepackage{booktabs}  % table support for pandoc > 1.12.2
    \usepackage[inline]{enumitem} % IRkernel/repr support (it uses the enumerate* environment)
    \usepackage[normalem]{ulem} % ulem is needed to support strikethroughs (\sout)
                                % normalem makes italics be italics, not underlines
    

    
    
    % Colors for the hyperref package
    \definecolor{urlcolor}{rgb}{0,.145,.698}
    \definecolor{linkcolor}{rgb}{.71,0.21,0.01}
    \definecolor{citecolor}{rgb}{.12,.54,.11}

    % ANSI colors
    \definecolor{ansi-black}{HTML}{3E424D}
    \definecolor{ansi-black-intense}{HTML}{282C36}
    \definecolor{ansi-red}{HTML}{E75C58}
    \definecolor{ansi-red-intense}{HTML}{B22B31}
    \definecolor{ansi-green}{HTML}{00A250}
    \definecolor{ansi-green-intense}{HTML}{007427}
    \definecolor{ansi-yellow}{HTML}{DDB62B}
    \definecolor{ansi-yellow-intense}{HTML}{B27D12}
    \definecolor{ansi-blue}{HTML}{208FFB}
    \definecolor{ansi-blue-intense}{HTML}{0065CA}
    \definecolor{ansi-magenta}{HTML}{D160C4}
    \definecolor{ansi-magenta-intense}{HTML}{A03196}
    \definecolor{ansi-cyan}{HTML}{60C6C8}
    \definecolor{ansi-cyan-intense}{HTML}{258F8F}
    \definecolor{ansi-white}{HTML}{C5C1B4}
    \definecolor{ansi-white-intense}{HTML}{A1A6B2}

    % commands and environments needed by pandoc snippets
    % extracted from the output of `pandoc -s`
    \providecommand{\tightlist}{%
      \setlength{\itemsep}{0pt}\setlength{\parskip}{0pt}}
    \DefineVerbatimEnvironment{Highlighting}{Verbatim}{commandchars=\\\{\}}
    % Add ',fontsize=\small' for more characters per line
    \newenvironment{Shaded}{}{}
    \newcommand{\KeywordTok}[1]{\textcolor[rgb]{0.00,0.44,0.13}{\textbf{{#1}}}}
    \newcommand{\DataTypeTok}[1]{\textcolor[rgb]{0.56,0.13,0.00}{{#1}}}
    \newcommand{\DecValTok}[1]{\textcolor[rgb]{0.25,0.63,0.44}{{#1}}}
    \newcommand{\BaseNTok}[1]{\textcolor[rgb]{0.25,0.63,0.44}{{#1}}}
    \newcommand{\FloatTok}[1]{\textcolor[rgb]{0.25,0.63,0.44}{{#1}}}
    \newcommand{\CharTok}[1]{\textcolor[rgb]{0.25,0.44,0.63}{{#1}}}
    \newcommand{\StringTok}[1]{\textcolor[rgb]{0.25,0.44,0.63}{{#1}}}
    \newcommand{\CommentTok}[1]{\textcolor[rgb]{0.38,0.63,0.69}{\textit{{#1}}}}
    \newcommand{\OtherTok}[1]{\textcolor[rgb]{0.00,0.44,0.13}{{#1}}}
    \newcommand{\AlertTok}[1]{\textcolor[rgb]{1.00,0.00,0.00}{\textbf{{#1}}}}
    \newcommand{\FunctionTok}[1]{\textcolor[rgb]{0.02,0.16,0.49}{{#1}}}
    \newcommand{\RegionMarkerTok}[1]{{#1}}
    \newcommand{\ErrorTok}[1]{\textcolor[rgb]{1.00,0.00,0.00}{\textbf{{#1}}}}
    \newcommand{\NormalTok}[1]{{#1}}
    
    % Additional commands for more recent versions of Pandoc
    \newcommand{\ConstantTok}[1]{\textcolor[rgb]{0.53,0.00,0.00}{{#1}}}
    \newcommand{\SpecialCharTok}[1]{\textcolor[rgb]{0.25,0.44,0.63}{{#1}}}
    \newcommand{\VerbatimStringTok}[1]{\textcolor[rgb]{0.25,0.44,0.63}{{#1}}}
    \newcommand{\SpecialStringTok}[1]{\textcolor[rgb]{0.73,0.40,0.53}{{#1}}}
    \newcommand{\ImportTok}[1]{{#1}}
    \newcommand{\DocumentationTok}[1]{\textcolor[rgb]{0.73,0.13,0.13}{\textit{{#1}}}}
    \newcommand{\AnnotationTok}[1]{\textcolor[rgb]{0.38,0.63,0.69}{\textbf{\textit{{#1}}}}}
    \newcommand{\CommentVarTok}[1]{\textcolor[rgb]{0.38,0.63,0.69}{\textbf{\textit{{#1}}}}}
    \newcommand{\VariableTok}[1]{\textcolor[rgb]{0.10,0.09,0.49}{{#1}}}
    \newcommand{\ControlFlowTok}[1]{\textcolor[rgb]{0.00,0.44,0.13}{\textbf{{#1}}}}
    \newcommand{\OperatorTok}[1]{\textcolor[rgb]{0.40,0.40,0.40}{{#1}}}
    \newcommand{\BuiltInTok}[1]{{#1}}
    \newcommand{\ExtensionTok}[1]{{#1}}
    \newcommand{\PreprocessorTok}[1]{\textcolor[rgb]{0.74,0.48,0.00}{{#1}}}
    \newcommand{\AttributeTok}[1]{\textcolor[rgb]{0.49,0.56,0.16}{{#1}}}
    \newcommand{\InformationTok}[1]{\textcolor[rgb]{0.38,0.63,0.69}{\textbf{\textit{{#1}}}}}
    \newcommand{\WarningTok}[1]{\textcolor[rgb]{0.38,0.63,0.69}{\textbf{\textit{{#1}}}}}
    
    
    % Define a nice break command that doesn't care if a line doesn't already
    % exist.
    \def\br{\hspace*{\fill} \\* }
    % Math Jax compatability definitions
    \def\gt{>}
    \def\lt{<}
    % Document parameters
    \title{Homework\_1}
    
    
    

    % Pygments definitions
    
\makeatletter
\def\PY@reset{\let\PY@it=\relax \let\PY@bf=\relax%
    \let\PY@ul=\relax \let\PY@tc=\relax%
    \let\PY@bc=\relax \let\PY@ff=\relax}
\def\PY@tok#1{\csname PY@tok@#1\endcsname}
\def\PY@toks#1+{\ifx\relax#1\empty\else%
    \PY@tok{#1}\expandafter\PY@toks\fi}
\def\PY@do#1{\PY@bc{\PY@tc{\PY@ul{%
    \PY@it{\PY@bf{\PY@ff{#1}}}}}}}
\def\PY#1#2{\PY@reset\PY@toks#1+\relax+\PY@do{#2}}

\expandafter\def\csname PY@tok@w\endcsname{\def\PY@tc##1{\textcolor[rgb]{0.73,0.73,0.73}{##1}}}
\expandafter\def\csname PY@tok@c\endcsname{\let\PY@it=\textit\def\PY@tc##1{\textcolor[rgb]{0.25,0.50,0.50}{##1}}}
\expandafter\def\csname PY@tok@cp\endcsname{\def\PY@tc##1{\textcolor[rgb]{0.74,0.48,0.00}{##1}}}
\expandafter\def\csname PY@tok@k\endcsname{\let\PY@bf=\textbf\def\PY@tc##1{\textcolor[rgb]{0.00,0.50,0.00}{##1}}}
\expandafter\def\csname PY@tok@kp\endcsname{\def\PY@tc##1{\textcolor[rgb]{0.00,0.50,0.00}{##1}}}
\expandafter\def\csname PY@tok@kt\endcsname{\def\PY@tc##1{\textcolor[rgb]{0.69,0.00,0.25}{##1}}}
\expandafter\def\csname PY@tok@o\endcsname{\def\PY@tc##1{\textcolor[rgb]{0.40,0.40,0.40}{##1}}}
\expandafter\def\csname PY@tok@ow\endcsname{\let\PY@bf=\textbf\def\PY@tc##1{\textcolor[rgb]{0.67,0.13,1.00}{##1}}}
\expandafter\def\csname PY@tok@nb\endcsname{\def\PY@tc##1{\textcolor[rgb]{0.00,0.50,0.00}{##1}}}
\expandafter\def\csname PY@tok@nf\endcsname{\def\PY@tc##1{\textcolor[rgb]{0.00,0.00,1.00}{##1}}}
\expandafter\def\csname PY@tok@nc\endcsname{\let\PY@bf=\textbf\def\PY@tc##1{\textcolor[rgb]{0.00,0.00,1.00}{##1}}}
\expandafter\def\csname PY@tok@nn\endcsname{\let\PY@bf=\textbf\def\PY@tc##1{\textcolor[rgb]{0.00,0.00,1.00}{##1}}}
\expandafter\def\csname PY@tok@ne\endcsname{\let\PY@bf=\textbf\def\PY@tc##1{\textcolor[rgb]{0.82,0.25,0.23}{##1}}}
\expandafter\def\csname PY@tok@nv\endcsname{\def\PY@tc##1{\textcolor[rgb]{0.10,0.09,0.49}{##1}}}
\expandafter\def\csname PY@tok@no\endcsname{\def\PY@tc##1{\textcolor[rgb]{0.53,0.00,0.00}{##1}}}
\expandafter\def\csname PY@tok@nl\endcsname{\def\PY@tc##1{\textcolor[rgb]{0.63,0.63,0.00}{##1}}}
\expandafter\def\csname PY@tok@ni\endcsname{\let\PY@bf=\textbf\def\PY@tc##1{\textcolor[rgb]{0.60,0.60,0.60}{##1}}}
\expandafter\def\csname PY@tok@na\endcsname{\def\PY@tc##1{\textcolor[rgb]{0.49,0.56,0.16}{##1}}}
\expandafter\def\csname PY@tok@nt\endcsname{\let\PY@bf=\textbf\def\PY@tc##1{\textcolor[rgb]{0.00,0.50,0.00}{##1}}}
\expandafter\def\csname PY@tok@nd\endcsname{\def\PY@tc##1{\textcolor[rgb]{0.67,0.13,1.00}{##1}}}
\expandafter\def\csname PY@tok@s\endcsname{\def\PY@tc##1{\textcolor[rgb]{0.73,0.13,0.13}{##1}}}
\expandafter\def\csname PY@tok@sd\endcsname{\let\PY@it=\textit\def\PY@tc##1{\textcolor[rgb]{0.73,0.13,0.13}{##1}}}
\expandafter\def\csname PY@tok@si\endcsname{\let\PY@bf=\textbf\def\PY@tc##1{\textcolor[rgb]{0.73,0.40,0.53}{##1}}}
\expandafter\def\csname PY@tok@se\endcsname{\let\PY@bf=\textbf\def\PY@tc##1{\textcolor[rgb]{0.73,0.40,0.13}{##1}}}
\expandafter\def\csname PY@tok@sr\endcsname{\def\PY@tc##1{\textcolor[rgb]{0.73,0.40,0.53}{##1}}}
\expandafter\def\csname PY@tok@ss\endcsname{\def\PY@tc##1{\textcolor[rgb]{0.10,0.09,0.49}{##1}}}
\expandafter\def\csname PY@tok@sx\endcsname{\def\PY@tc##1{\textcolor[rgb]{0.00,0.50,0.00}{##1}}}
\expandafter\def\csname PY@tok@m\endcsname{\def\PY@tc##1{\textcolor[rgb]{0.40,0.40,0.40}{##1}}}
\expandafter\def\csname PY@tok@gh\endcsname{\let\PY@bf=\textbf\def\PY@tc##1{\textcolor[rgb]{0.00,0.00,0.50}{##1}}}
\expandafter\def\csname PY@tok@gu\endcsname{\let\PY@bf=\textbf\def\PY@tc##1{\textcolor[rgb]{0.50,0.00,0.50}{##1}}}
\expandafter\def\csname PY@tok@gd\endcsname{\def\PY@tc##1{\textcolor[rgb]{0.63,0.00,0.00}{##1}}}
\expandafter\def\csname PY@tok@gi\endcsname{\def\PY@tc##1{\textcolor[rgb]{0.00,0.63,0.00}{##1}}}
\expandafter\def\csname PY@tok@gr\endcsname{\def\PY@tc##1{\textcolor[rgb]{1.00,0.00,0.00}{##1}}}
\expandafter\def\csname PY@tok@ge\endcsname{\let\PY@it=\textit}
\expandafter\def\csname PY@tok@gs\endcsname{\let\PY@bf=\textbf}
\expandafter\def\csname PY@tok@gp\endcsname{\let\PY@bf=\textbf\def\PY@tc##1{\textcolor[rgb]{0.00,0.00,0.50}{##1}}}
\expandafter\def\csname PY@tok@go\endcsname{\def\PY@tc##1{\textcolor[rgb]{0.53,0.53,0.53}{##1}}}
\expandafter\def\csname PY@tok@gt\endcsname{\def\PY@tc##1{\textcolor[rgb]{0.00,0.27,0.87}{##1}}}
\expandafter\def\csname PY@tok@err\endcsname{\def\PY@bc##1{\setlength{\fboxsep}{0pt}\fcolorbox[rgb]{1.00,0.00,0.00}{1,1,1}{\strut ##1}}}
\expandafter\def\csname PY@tok@kc\endcsname{\let\PY@bf=\textbf\def\PY@tc##1{\textcolor[rgb]{0.00,0.50,0.00}{##1}}}
\expandafter\def\csname PY@tok@kd\endcsname{\let\PY@bf=\textbf\def\PY@tc##1{\textcolor[rgb]{0.00,0.50,0.00}{##1}}}
\expandafter\def\csname PY@tok@kn\endcsname{\let\PY@bf=\textbf\def\PY@tc##1{\textcolor[rgb]{0.00,0.50,0.00}{##1}}}
\expandafter\def\csname PY@tok@kr\endcsname{\let\PY@bf=\textbf\def\PY@tc##1{\textcolor[rgb]{0.00,0.50,0.00}{##1}}}
\expandafter\def\csname PY@tok@bp\endcsname{\def\PY@tc##1{\textcolor[rgb]{0.00,0.50,0.00}{##1}}}
\expandafter\def\csname PY@tok@fm\endcsname{\def\PY@tc##1{\textcolor[rgb]{0.00,0.00,1.00}{##1}}}
\expandafter\def\csname PY@tok@vc\endcsname{\def\PY@tc##1{\textcolor[rgb]{0.10,0.09,0.49}{##1}}}
\expandafter\def\csname PY@tok@vg\endcsname{\def\PY@tc##1{\textcolor[rgb]{0.10,0.09,0.49}{##1}}}
\expandafter\def\csname PY@tok@vi\endcsname{\def\PY@tc##1{\textcolor[rgb]{0.10,0.09,0.49}{##1}}}
\expandafter\def\csname PY@tok@vm\endcsname{\def\PY@tc##1{\textcolor[rgb]{0.10,0.09,0.49}{##1}}}
\expandafter\def\csname PY@tok@sa\endcsname{\def\PY@tc##1{\textcolor[rgb]{0.73,0.13,0.13}{##1}}}
\expandafter\def\csname PY@tok@sb\endcsname{\def\PY@tc##1{\textcolor[rgb]{0.73,0.13,0.13}{##1}}}
\expandafter\def\csname PY@tok@sc\endcsname{\def\PY@tc##1{\textcolor[rgb]{0.73,0.13,0.13}{##1}}}
\expandafter\def\csname PY@tok@dl\endcsname{\def\PY@tc##1{\textcolor[rgb]{0.73,0.13,0.13}{##1}}}
\expandafter\def\csname PY@tok@s2\endcsname{\def\PY@tc##1{\textcolor[rgb]{0.73,0.13,0.13}{##1}}}
\expandafter\def\csname PY@tok@sh\endcsname{\def\PY@tc##1{\textcolor[rgb]{0.73,0.13,0.13}{##1}}}
\expandafter\def\csname PY@tok@s1\endcsname{\def\PY@tc##1{\textcolor[rgb]{0.73,0.13,0.13}{##1}}}
\expandafter\def\csname PY@tok@mb\endcsname{\def\PY@tc##1{\textcolor[rgb]{0.40,0.40,0.40}{##1}}}
\expandafter\def\csname PY@tok@mf\endcsname{\def\PY@tc##1{\textcolor[rgb]{0.40,0.40,0.40}{##1}}}
\expandafter\def\csname PY@tok@mh\endcsname{\def\PY@tc##1{\textcolor[rgb]{0.40,0.40,0.40}{##1}}}
\expandafter\def\csname PY@tok@mi\endcsname{\def\PY@tc##1{\textcolor[rgb]{0.40,0.40,0.40}{##1}}}
\expandafter\def\csname PY@tok@il\endcsname{\def\PY@tc##1{\textcolor[rgb]{0.40,0.40,0.40}{##1}}}
\expandafter\def\csname PY@tok@mo\endcsname{\def\PY@tc##1{\textcolor[rgb]{0.40,0.40,0.40}{##1}}}
\expandafter\def\csname PY@tok@ch\endcsname{\let\PY@it=\textit\def\PY@tc##1{\textcolor[rgb]{0.25,0.50,0.50}{##1}}}
\expandafter\def\csname PY@tok@cm\endcsname{\let\PY@it=\textit\def\PY@tc##1{\textcolor[rgb]{0.25,0.50,0.50}{##1}}}
\expandafter\def\csname PY@tok@cpf\endcsname{\let\PY@it=\textit\def\PY@tc##1{\textcolor[rgb]{0.25,0.50,0.50}{##1}}}
\expandafter\def\csname PY@tok@c1\endcsname{\let\PY@it=\textit\def\PY@tc##1{\textcolor[rgb]{0.25,0.50,0.50}{##1}}}
\expandafter\def\csname PY@tok@cs\endcsname{\let\PY@it=\textit\def\PY@tc##1{\textcolor[rgb]{0.25,0.50,0.50}{##1}}}

\def\PYZbs{\char`\\}
\def\PYZus{\char`\_}
\def\PYZob{\char`\{}
\def\PYZcb{\char`\}}
\def\PYZca{\char`\^}
\def\PYZam{\char`\&}
\def\PYZlt{\char`\<}
\def\PYZgt{\char`\>}
\def\PYZsh{\char`\#}
\def\PYZpc{\char`\%}
\def\PYZdl{\char`\$}
\def\PYZhy{\char`\-}
\def\PYZsq{\char`\'}
\def\PYZdq{\char`\"}
\def\PYZti{\char`\~}
% for compatibility with earlier versions
\def\PYZat{@}
\def\PYZlb{[}
\def\PYZrb{]}
\makeatother


    % Exact colors from NB
    \definecolor{incolor}{rgb}{0.0, 0.0, 0.5}
    \definecolor{outcolor}{rgb}{0.545, 0.0, 0.0}



    
    % Prevent overflowing lines due to hard-to-break entities
    \sloppy 
    % Setup hyperref package
    \hypersetup{
      breaklinks=true,  % so long urls are correctly broken across lines
      colorlinks=true,
      urlcolor=urlcolor,
      linkcolor=linkcolor,
      citecolor=citecolor,
      }
    % Slightly bigger margins than the latex defaults
    
    \geometry{verbose,tmargin=1in,bmargin=1in,lmargin=1in,rmargin=1in}
    
    

    \begin{document}
 
\noindent \LARGE{Homework 1}

\noindent \large{Arnold Leigh Ryan Choa --- 32038144}

\noindent \large{PHYS 410}

\noindent \large{Summary of Answers are at the last page}

    \begin{Verbatim}[commandchars=\\\{\}]
{\color{incolor}In [{\color{incolor}1}]:} \PY{k+kn}{import} \PY{n+nn}{numpy} \PY{k}{as} \PY{n+nn}{np}
        \PY{k+kn}{import} \PY{n+nn}{matplotlib}\PY{n+nn}{.}\PY{n+nn}{pyplot} \PY{k}{as} \PY{n+nn}{plt}
        \PY{k+kn}{import} \PY{n+nn}{pandas} \PY{k}{as} \PY{n+nn}{pd} \PY{c+c1}{\PYZsh{}for tabling results}
\end{Verbatim}

\noindent \Large{Question 1}
\\ \\
\noindent \large{Functions for Simple Bisection and 1-D Newton's Method:}

    \begin{Verbatim}[commandchars=\\\{\}]
{\color{incolor}In [{\color{incolor}2}]:} \PY{l+s+sd}{\PYZdq{}\PYZdq{}\PYZdq{}}
        \PY{l+s+sd}{Code for a simple bisection}
        \PY{l+s+sd}{[a,b] signifies the starting interval.}
        \PY{l+s+sd}{givenFunction is the function of the sepcific problem we want}
        \PY{l+s+sd}{delta is a threshold value to function(x) for saying approximation}
        \PY{l+s+sd}{is good enough}
        \PY{l+s+sd}{\PYZdq{}\PYZdq{}\PYZdq{}}
        \PY{k}{def} \PY{n+nf}{bisection}\PY{p}{(}\PY{n}{a}\PY{p}{,}\PY{n}{b}\PY{p}{,} \PY{n}{givenFunction}\PY{p}{,} \PY{n}{delta} \PY{o}{=} \PY{l+m+mf}{0.01}\PY{p}{)}\PY{p}{:}
            \PY{k}{if} \PY{n}{a}\PY{o}{\PYZgt{}}\PY{n}{b}\PY{p}{:}
                \PY{c+c1}{\PYZsh{} Checking if a \PYZlt{}= b}
                \PY{k}{return} \PY{n}{bisection}\PY{p}{(}\PY{n}{b}\PY{p}{,}\PY{n}{a}\PY{p}{,} \PY{n}{givenFunction}\PY{p}{,} \PY{n}{delta}\PY{p}{)}
            \PY{n}{mid} \PY{o}{=} \PY{p}{(}\PY{n}{a}\PY{o}{+}\PY{n}{b}\PY{p}{)}\PY{o}{/}\PY{l+m+mi}{2}
            \PY{k}{if} \PY{n+nb}{abs}\PY{p}{(}\PY{n}{givenFunction}\PY{p}{(}\PY{n}{mid}\PY{p}{)}\PY{p}{)} \PY{o}{\PYZlt{}} \PY{n}{delta}\PY{p}{:} \PY{c+c1}{\PYZsh{} Test delta against value of root}
                \PY{k}{return} \PY{n}{mid} \PY{c+c1}{\PYZsh{} Return midpoint of interval if bisection is  satisfied}
            \PY{k}{else}\PY{p}{:}
                \PY{k}{if} \PY{n}{givenFunction}\PY{p}{(}\PY{n}{a}\PY{p}{)}\PY{o}{*}\PY{n}{givenFunction}\PY{p}{(}\PY{n}{mid}\PY{p}{)} \PY{o}{\PYZlt{}} \PY{l+m+mi}{0}\PY{p}{:}
                    \PY{k}{return} \PY{n}{bisection}\PY{p}{(}\PY{n}{a}\PY{p}{,} \PY{n}{mid}\PY{p}{,} \PY{n}{givenFunction}\PY{p}{,} \PY{n}{delta}\PY{p}{)}
                \PY{k}{else}\PY{p}{:}
                    \PY{k}{return} \PY{n}{bisection}\PY{p}{(}\PY{n}{mid}\PY{p}{,} \PY{n}{b}\PY{p}{,} \PY{n}{givenFunction}\PY{p}{,} \PY{n}{delta}\PY{p}{)}
\end{Verbatim}

    \begin{Verbatim}[commandchars=\\\{\}]
{\color{incolor}In [{\color{incolor}16}]:} \PY{l+s+sd}{\PYZdq{}\PYZdq{}\PYZdq{}}
         \PY{l+s+sd}{Code for a 1\PYZhy{}D Newton\PYZsq{}s method}
         \PY{l+s+sd}{x signifies the starting point.}
         \PY{l+s+sd}{givenFunction is the function of the sepcific problem we want}
         \PY{l+s+sd}{givenDerivative is the function of the sepcific problem we want}
         \PY{l+s+sd}{delta is a threshold value to function(x) for saying approximation}
         \PY{l+s+sd}{is good enough}
         \PY{l+s+sd}{a is the learning rate}
         \PY{l+s+sd}{\PYZdq{}\PYZdq{}\PYZdq{}}
         \PY{k}{def} \PY{n+nf}{one\PYZus{}Newton}\PY{p}{(}\PY{n}{x}\PY{p}{,} \PY{n}{givenFunction}\PY{p}{,} \PY{n}{givenDerivative}\PY{p}{,} \PY{n}{delta}\PY{p}{,} \PY{n}{a}\PY{p}{)}\PY{p}{:}
             \PY{n}{value} \PY{o}{=} \PY{n}{givenFunction}\PY{p}{(}\PY{n}{x}\PY{p}{)}
             \PY{n}{slope} \PY{o}{=} \PY{n}{givenDerivative}\PY{p}{(}\PY{n}{x}\PY{p}{)}
             
             \PY{k}{if} \PY{n+nb}{abs}\PY{p}{(}\PY{n}{value}\PY{p}{)} \PY{o}{\PYZlt{}} \PY{n}{delta}\PY{p}{:}
                 \PY{k}{return} \PY{n}{x}
             \PY{k}{else}\PY{p}{:}
                 \PY{n}{xNew} \PY{o}{=} \PY{n}{x} \PY{o}{\PYZhy{}} \PY{p}{(}\PY{n}{value}\PY{o}{/}\PY{n}{slope}\PY{p}{)}
                 \PY{k}{if} \PY{n+nb}{abs}\PY{p}{(}\PY{n}{xNew} \PY{o}{\PYZhy{}} \PY{n}{x}\PY{p}{)} \PY{o}{\PYZlt{}} \PY{n}{a}\PY{p}{:}
                     \PY{n}{xNew} \PY{o}{=} \PY{n}{x} \PY{o}{\PYZhy{}} \PY{n}{a}\PY{o}{*}\PY{n}{np}\PY{o}{.}\PY{n}{sign}\PY{p}{(}\PY{n}{value}\PY{o}{/}\PY{n}{slope}\PY{p}{)}
                 \PY{k}{return} \PY{n}{one\PYZus{}Newton}\PY{p}{(}\PY{n}{xNew}\PY{p}{,} \PY{n}{givenFunction}\PY{p}{,} \PY{n}{givenDerivative}\PY{p}{,} \PY{n}{delta}\PY{p}{,} \PY{n}{a}\PY{p}{)}
\end{Verbatim}

\noindent \large{Function and Derivative for Question 1:}

    \begin{Verbatim}[commandchars=\\\{\}]
{\color{incolor}In [{\color{incolor}3}]:} \PY{k}{def} \PY{n+nf}{q1Function}\PY{p}{(}\PY{n}{x}\PY{p}{)}\PY{p}{:}
            \PY{k}{return}\PY{p}{(}\PY{l+m+mi}{6425}\PY{o}{*}\PY{n}{x}\PY{o}{*}\PY{o}{*}\PY{l+m+mi}{8} \PY{o}{\PYZhy{}} \PY{l+m+mi}{12012}\PY{o}{*}\PY{n}{x}\PY{o}{*}\PY{o}{*}\PY{l+m+mi}{6} \PY{o}{+} \PY{l+m+mi}{6930}\PY{o}{*}\PY{n}{x}\PY{o}{*}\PY{o}{*}\PY{l+m+mi}{4} \PY{o}{\PYZhy{}} \PY{l+m+mi}{1260}\PY{o}{*}\PY{n}{x}\PY{o}{*}\PY{o}{*}\PY{l+m+mi}{2} \PY{o}{+} \PY{l+m+mi}{35}\PY{p}{)}\PY{o}{/}\PY{l+m+mi}{128}
            
        \PY{k}{def} \PY{n+nf}{q1Derivative}\PY{p}{(}\PY{n}{x}\PY{p}{)}\PY{p}{:}
            \PY{k}{return}\PY{p}{(}\PY{l+m+mi}{8}\PY{o}{*}\PY{l+m+mi}{6425}\PY{o}{*}\PY{n}{x}\PY{o}{*}\PY{o}{*}\PY{l+m+mi}{7} \PY{o}{\PYZhy{}} \PY{l+m+mi}{6}\PY{o}{*}\PY{l+m+mi}{12012}\PY{o}{*}\PY{n}{x}\PY{o}{*}\PY{o}{*}\PY{l+m+mi}{5} \PY{o}{+} \PY{l+m+mi}{4}\PY{o}{*}\PY{l+m+mi}{6930}\PY{o}{*}\PY{n}{x}\PY{o}{*}\PY{o}{*}\PY{l+m+mi}{3} \PY{o}{\PYZhy{}} \PY{l+m+mi}{2}\PY{o}{*}\PY{l+m+mi}{1260}\PY{o}{*}\PY{n}{x}\PY{p}{)}\PY{o}{/}\PY{l+m+mi}{128}
\end{Verbatim}

    \begin{Verbatim}[commandchars=\\\{\}]
{\color{incolor}In [{\color{incolor}4}]:} \PY{n}{x} \PY{o}{=} \PY{n}{np}\PY{o}{.}\PY{n}{linspace}\PY{p}{(}\PY{o}{\PYZhy{}}\PY{l+m+mi}{1}\PY{p}{,}\PY{l+m+mi}{1}\PY{p}{,}\PY{l+m+mi}{1000}\PY{p}{)}
        \PY{n}{y} \PY{o}{=} \PY{n}{np}\PY{o}{.}\PY{n}{apply\PYZus{}along\PYZus{}axis}\PY{p}{(}\PY{n}{q1Function}\PY{p}{,}\PY{l+m+mi}{0}\PY{p}{,} \PY{n}{x}\PY{p}{)}
        \PY{n}{y2} \PY{o}{=} \PY{n}{np}\PY{o}{.}\PY{n}{zeros}\PY{p}{(}\PY{n+nb}{len}\PY{p}{(}\PY{n}{x}\PY{p}{)}\PY{p}{)}
        
        \PY{n}{plt}\PY{o}{.}\PY{n}{figure}\PY{p}{(}\PY{l+m+mi}{1}\PY{p}{)}
        \PY{n}{plt}\PY{o}{.}\PY{n}{plot}\PY{p}{(}\PY{n}{x}\PY{p}{,}\PY{n}{y}\PY{p}{,} \PY{n}{label} \PY{o}{=} \PY{l+s+s2}{\PYZdq{}}\PY{l+s+s2}{F(x)}\PY{l+s+s2}{\PYZdq{}}\PY{p}{)}
        \PY{n}{plt}\PY{o}{.}\PY{n}{plot}\PY{p}{(}\PY{n}{x}\PY{p}{,}\PY{n}{y2}\PY{p}{,} \PY{n}{label} \PY{o}{=} \PY{l+s+s2}{\PYZdq{}}\PY{l+s+s2}{y=0}\PY{l+s+s2}{\PYZdq{}}\PY{p}{)}
        \PY{n}{plt}\PY{o}{.}\PY{n}{xlabel}\PY{p}{(}\PY{l+s+s2}{\PYZdq{}}\PY{l+s+s2}{X}\PY{l+s+s2}{\PYZdq{}}\PY{p}{)}
        \PY{n}{plt}\PY{o}{.}\PY{n}{ylabel}\PY{p}{(}\PY{l+s+s2}{\PYZdq{}}\PY{l+s+s2}{Y}\PY{l+s+s2}{\PYZdq{}}\PY{p}{)}
        \PY{n}{plt}\PY{o}{.}\PY{n}{legend}\PY{p}{(}\PY{p}{)}
        \PY{n}{plt}\PY{o}{.}\PY{n}{title}\PY{p}{(}\PY{l+s+s2}{\PYZdq{}}\PY{l+s+s2}{Plot of Question 1}\PY{l+s+s2}{\PYZdq{}}\PY{p}{)}
        \PY{n}{plt}\PY{o}{.}\PY{n}{show}\PY{p}{(}\PY{p}{)}
\end{Verbatim}

    \begin{center}
    \adjustimage{max size={0.9\linewidth}{0.9\paperheight}}{output_8_0.png}
    \end{center}
    { \hspace*{\fill} \\}
    
\noindent By plotting, we now that there are 8 single roots in between \([-1,1]\).
We also now, by observation, that there are exactly one root in these
intervals:
\(\{[-1,-0.90],[-0.90,-0.60],[-0.60,-0.30],[-0.30,0.0],[0.0,0.30],[0.30,0.60],[0.60,0.90],[0.90,1.0]\}\)
\\ \\
\noindent We want to use a hybrid method of Bisection and Newton. To do that, we
first have to write a function that hybridizes them:

    \begin{Verbatim}[commandchars=\\\{\}]
{\color{incolor}In [{\color{incolor}5}]:} \PY{l+s+sd}{\PYZdq{}\PYZdq{}\PYZdq{}}
        \PY{l+s+sd}{hybrid hybridizes the bisection and 1D Newtion method}
        \PY{l+s+sd}{[a,b] signify the starting interval.}
        \PY{l+s+sd}{givenFunction is the function of the sepcific problem we want}
        \PY{l+s+sd}{givenDerivative is the function of the sepcific problem we want}
        \PY{l+s+sd}{bis\PYZus{}delta is a threshold value to function(x) for saying bisection}
        \PY{l+s+sd}{is good enough}
        \PY{l+s+sd}{newton\PYZus{}delta is a threshold value to function(x) for saying bisection}
        \PY{l+s+sd}{is good enough}
        \PY{l+s+sd}{newton\PYZus{}a is the max learning rate for the 1D newton}
        \PY{l+s+sd}{\PYZdq{}\PYZdq{}\PYZdq{}}
        \PY{k}{def} \PY{n+nf}{hybrid}\PY{p}{(}\PY{n}{a}\PY{p}{,}\PY{n}{b}\PY{p}{,} \PY{n}{bis\PYZus{}delta}\PY{p}{,} \PY{n}{newton\PYZus{}delta}\PY{p}{,} \PY{n}{newton\PYZus{}a}\PY{p}{,} \PY{n}{givenFunction}\PY{p}{,}
\hspace{4cm} \PY{n}{givenDerivative}\PY{p}{)}\PY{p}{:}
            \PY{n}{x0} \PY{o}{=} \PY{n}{bisection}\PY{p}{(}\PY{n}{a}\PY{p}{,}\PY{n}{b}\PY{p}{,}\PY{n}{givenFunction}\PY{p}{,} \PY{n}{bis\PYZus{}delta}\PY{p}{)}
            \PY{n}{root} \PY{o}{=} \PY{n}{one\PYZus{}Newton}\PY{p}{(}\PY{n}{x0}\PY{p}{,}\PY{n}{givenFunction}\PY{p}{,} \PY{n}{givenDerivative}\PY{p}{,} \PY{n}{newton\PYZus{}delta}\PY{p}{,}
\hspace{7cm} \PY{n}{newton\PYZus{}a}\PY{p}{)}
            \PY{k}{return} \PY{n}{root}
\end{Verbatim}

    Now, calculating approximates roots with threshold deltas:

    \begin{Verbatim}[commandchars=\\\{\}]
{\color{incolor}In [{\color{incolor}6}]:} \PY{n}{cols} \PY{o}{=} \PY{p}{[}\PY{l+s+s2}{\PYZdq{}}\PY{l+s+s2}{Interval Start}\PY{l+s+s2}{\PYZdq{}}\PY{p}{,} \PY{l+s+s2}{\PYZdq{}}\PY{l+s+s2}{Interval End}\PY{l+s+s2}{\PYZdq{}}\PY{p}{,} \PY{l+s+s2}{\PYZdq{}}\PY{l+s+s2}{Approximate Root}\PY{l+s+s2}{\PYZdq{}}\PY{p}{,} \PY{l+s+s2}{\PYZdq{}}\PY{l+s+s2}{Value of Approx}\PY{l+s+s2}{\PYZdq{}}\PY{p}{]}
        \PY{n}{data} \PY{o}{=} \PY{p}{[}\PY{p}{]}
        \PY{n}{candidates} \PY{o}{=} \PY{p}{[}\PY{p}{[}\PY{o}{\PYZhy{}}\PY{l+m+mi}{1}\PY{p}{,}\PY{o}{\PYZhy{}}\PY{l+m+mf}{0.90}\PY{p}{]}\PY{p}{,}\PY{p}{[}\PY{o}{\PYZhy{}}\PY{l+m+mf}{0.90}\PY{p}{,}\PY{o}{\PYZhy{}}\PY{l+m+mf}{0.60}\PY{p}{]}\PY{p}{,}\PY{p}{[}\PY{o}{\PYZhy{}}\PY{l+m+mf}{0.60}\PY{p}{,}\PY{o}{\PYZhy{}}\PY{l+m+mf}{0.30}\PY{p}{]}\PY{p}{,}\PY{p}{[}\PY{o}{\PYZhy{}}\PY{l+m+mf}{0.30}\PY{p}{,}\PY{l+m+mf}{0.0}\PY{p}{]}\PY{p}{,}
\hspace{5cm} \PY{p}{[}\PY{l+m+mf}{0.0}\PY{p}{,}\PY{l+m+mf}{0.30}\PY{p}{]}\PY{p}{,}\PY{p}{[}\PY{l+m+mf}{0.30}\PY{p}{,}\PY{l+m+mf}{0.60}\PY{p}{]}\PY{p}{,}\PY{p}{[}\PY{l+m+mf}{0.60}\PY{p}{,}\PY{l+m+mf}{0.90}\PY{p}{]}\PY{p}{,}\PY{p}{[}\PY{l+m+mf}{0.90}\PY{p}{,}\PY{l+m+mf}{1.0}\PY{p}{]}\PY{p}{]}
        \PY{n}{bisectionDelta} \PY{o}{=} \PY{l+m+mf}{0.001}
        \PY{n}{newtonDelta} \PY{o}{=} \PY{l+m+mf}{0.000001}
        \PY{n}{newtonA} \PY{o}{=} \PY{l+m+mf}{0.00000001}
        
        \PY{k}{for} \PY{n}{c} \PY{o+ow}{in} \PY{n}{candidates}\PY{p}{:}
            \PY{n}{a} \PY{o}{=} \PY{n}{c}\PY{p}{[}\PY{l+m+mi}{0}\PY{p}{]}
            \PY{n}{b} \PY{o}{=} \PY{n}{c}\PY{p}{[}\PY{l+m+mi}{1}\PY{p}{]}
            \PY{n}{root} \PY{o}{=} \PY{n}{hybrid}\PY{p}{(}\PY{n}{a}\PY{p}{,} \PY{n}{b}\PY{p}{,} \PY{n}{bisectionDelta}\PY{p}{,} \PY{n}{newtonDelta}\PY{p}{,} \PY{n}{newtonA}\PY{p}{,}
\hspace{5.5cm} \PY{n}{q1Function}\PY{p}{,} \PY{n}{q1Derivative}\PY{p}{)}
            \PY{n}{data}\PY{o}{.}\PY{n}{append}\PY{p}{(}\PY{p}{[}\PY{n}{c}\PY{p}{[}\PY{l+m+mi}{0}\PY{p}{]}\PY{p}{,}\PY{n}{c}\PY{p}{[}\PY{l+m+mi}{1}\PY{p}{]}\PY{p}{,}\PY{n}{root}\PY{p}{,} \PY{n}{q1Function}\PY{p}{(}\PY{n}{root}\PY{p}{)}\PY{p}{]}\PY{p}{)}
        
        \PY{n}{p} \PY{o}{=} \PY{n}{pd}\PY{o}{.}\PY{n}{DataFrame}\PY{p}{(}\PY{n}{data}\PY{p}{,} \PY{n}{columns}\PY{o}{=}\PY{n}{cols}\PY{p}{)}
        \PY{n}{p}
\end{Verbatim}

            \begin{Verbatim}[commandchars=\\\{\}]
{\color{outcolor}Out[{\color{outcolor}6}]:}    Interval Start  Interval End  Approximate Root  Value of Approx
        0            -1.0          -0.9         -0.963787     6.548784e-08
        1            -0.9          -0.6         -0.794161    -7.321224e-08
        2            -0.6          -0.3         -0.525686     2.095979e-08
        3            -0.3           0.0         -0.183435    -5.958460e-10
        4             0.0           0.3          0.183435    -5.958460e-10
        5             0.3           0.6          0.525686     2.095979e-08
        6             0.6           0.9          0.794161    -7.321224e-08
        7             0.9           1.0          0.963787     6.548784e-08
\end{Verbatim}
        
    Plotting the approximate roots on the graph:

    \begin{Verbatim}[commandchars=\\\{\}]
{\color{incolor}In [{\color{incolor}7}]:} \PY{n}{x} \PY{o}{=} \PY{n}{np}\PY{o}{.}\PY{n}{linspace}\PY{p}{(}\PY{o}{\PYZhy{}}\PY{l+m+mi}{1}\PY{p}{,}\PY{l+m+mi}{1}\PY{p}{,}\PY{l+m+mi}{1000}\PY{p}{)}
        \PY{n}{y} \PY{o}{=} \PY{n}{np}\PY{o}{.}\PY{n}{apply\PYZus{}along\PYZus{}axis}\PY{p}{(}\PY{n}{q1Function}\PY{p}{,}\PY{l+m+mi}{0}\PY{p}{,} \PY{n}{x}\PY{p}{)}
        \PY{n}{y2} \PY{o}{=} \PY{n}{np}\PY{o}{.}\PY{n}{zeros}\PY{p}{(}\PY{n+nb}{len}\PY{p}{(}\PY{n}{x}\PY{p}{)}\PY{p}{)}
        
        \PY{n}{plt}\PY{o}{.}\PY{n}{figure}\PY{p}{(}\PY{l+m+mi}{2}\PY{p}{)}
        \PY{n}{plt}\PY{o}{.}\PY{n}{plot}\PY{p}{(}\PY{n}{x}\PY{p}{,}\PY{n}{y}\PY{p}{,} \PY{n}{label} \PY{o}{=} \PY{l+s+s2}{\PYZdq{}}\PY{l+s+s2}{F(x)}\PY{l+s+s2}{\PYZdq{}}\PY{p}{)}
        \PY{n}{plt}\PY{o}{.}\PY{n}{plot}\PY{p}{(}\PY{n}{x}\PY{p}{,}\PY{n}{y2}\PY{p}{,} \PY{n}{label} \PY{o}{=} \PY{l+s+s2}{\PYZdq{}}\PY{l+s+s2}{y=0}\PY{l+s+s2}{\PYZdq{}}\PY{p}{)}
        \PY{n}{plt}\PY{o}{.}\PY{n}{scatter}\PY{p}{(}\PY{n}{p}\PY{p}{[}\PY{l+s+s2}{\PYZdq{}}\PY{l+s+s2}{Approximate Root}\PY{l+s+s2}{\PYZdq{}}\PY{p}{]}\PY{p}{,} \PY{n}{p}\PY{p}{[}\PY{l+s+s2}{\PYZdq{}}\PY{l+s+s2}{Value of Approx}\PY{l+s+s2}{\PYZdq{}}\PY{p}{]}\PY{p}{,} \PY{n}{label}\PY{o}{=}\PY{k+kc}{None}\PY{p}{)}
        \PY{n}{plt}\PY{o}{.}\PY{n}{xlabel}\PY{p}{(}\PY{l+s+s2}{\PYZdq{}}\PY{l+s+s2}{X}\PY{l+s+s2}{\PYZdq{}}\PY{p}{)}
        \PY{n}{plt}\PY{o}{.}\PY{n}{ylabel}\PY{p}{(}\PY{l+s+s2}{\PYZdq{}}\PY{l+s+s2}{Y}\PY{l+s+s2}{\PYZdq{}}\PY{p}{)}
        \PY{n}{plt}\PY{o}{.}\PY{n}{legend}\PY{p}{(}\PY{p}{)}
        \PY{n}{plt}\PY{o}{.}\PY{n}{title}\PY{p}{(}\PY{l+s+s2}{\PYZdq{}}\PY{l+s+s2}{Plot of Question 1 with Approximate Roots}\PY{l+s+s2}{\PYZdq{}}\PY{p}{)}
        \PY{n}{plt}\PY{o}{.}\PY{n}{show}\PY{p}{(}\PY{p}{)}
\end{Verbatim}

    \begin{center}
    \adjustimage{max size={0.9\linewidth}{0.9\paperheight}}{output_15_0.png}
    \end{center}
    { \hspace*{\fill} \\}
    
\noindent \Large{Question 2}

\noindent \large{Function for n-D Newton's Method:}

    \begin{Verbatim}[commandchars=\\\{\}]
{\color{incolor}In [{\color{incolor}8}]:} \PY{l+s+sd}{\PYZdq{}\PYZdq{}\PYZdq{}}
        \PY{l+s+sd}{Code for a n\PYZhy{}D Newton\PYZsq{}s method}
        \PY{l+s+sd}{X signifies the starting point. This is a 1xn array}
        \PY{l+s+sd}{givenFunctions is an array of functions of the sepcific problem we want.}
        \PY{l+s+sd}{ Should be a 1xn array.}
        \PY{l+s+sd}{givenDerivatives is a Jacobian matrix of partial derivatives of the}
        \PY{l+s+sd}{ sepcific problem we want. Should an nxn matrix}
        \PY{l+s+sd}{delta is a threshold error for saying approximation is good enough.}
        \PY{l+s+sd}{ All functions must fall within threshold.}
        \PY{l+s+sd}{\PYZdq{}\PYZdq{}\PYZdq{}}
        \PY{k}{def} \PY{n+nf}{n\PYZus{}Newton}\PY{p}{(}\PY{n}{X}\PY{p}{,} \PY{n}{givenFunctions}\PY{p}{,} \PY{n}{givenDerivatives}\PY{p}{,} \PY{n}{delta}\PY{p}{)}\PY{p}{:}
	 \PY{c+c1}{\PYZsh{} Values of each function at X}
            \PY{n}{b} \PY{o}{=} \PY{n}{np}\PY{o}{.}\PY{n}{array}\PY{p}{(}\PY{p}{[}\PY{n}{g}\PY{p}{(}\PY{n}{X}\PY{p}{)} \PY{k}{for} \PY{n}{g} \PY{o+ow}{in} \PY{n}{givenFunctions}\PY{p}{]}\PY{p}{)}
	 \PY{c+c1}{\PYZsh{} Jacobian Matrix evaluated at X}
            \PY{n}{J} \PY{o}{=} \PY{n}{np}\PY{o}{.}\PY{n}{array}\PY{p}{(}\PY{p}{[}\PY{n}{np}\PY{o}{.}\PY{n}{array}\PY{p}{(}\PY{p}{[}\PY{n}{g}\PY{p}{(}\PY{n}{X}\PY{p}{)} \PY{k}{for} \PY{n}{g} \PY{o+ow}{in} \PY{n}{G}\PY{p}{]}\PY{p}{)} \PY{k}{for} \PY{n}{G} \PY{o+ow}{in} \PY{n}{givenDerivatives}\PY{p}{]}\PY{p}{)} 
            
            \PY{k}{if} \PY{n+nb}{all}\PY{p}{(}\PY{p}{[}\PY{n+nb}{abs}\PY{p}{(}\PY{n}{x}\PY{p}{)} \PY{o}{\PYZlt{}} \PY{n}{delta} \PY{k}{for} \PY{n}{x} \PY{o+ow}{in} \PY{n}{b}\PY{p}{]}\PY{p}{)}\PY{p}{:}
                \PY{k}{return} \PY{n}{X}
            \PY{k}{else}\PY{p}{:}
                \PY{n}{dr} \PY{o}{=} \PY{n}{np}\PY{o}{.}\PY{n}{linalg}\PY{o}{.}\PY{n}{solve}\PY{p}{(}\PY{n}{J}\PY{p}{,} \PY{o}{\PYZhy{}}\PY{n}{b}\PY{p}{)} \PY{c+c1}{\PYZsh{}Python\PYZsq{}s answer to MATLAB\PYZsq{}s linsolve}
                \PY{n}{xNew} \PY{o}{=} \PY{n}{X} \PY{o}{+} \PY{n}{dr}
                \PY{k}{return} \PY{n}{n\PYZus{}Newton}\PY{p}{(}\PY{n}{xNew}\PY{p}{,} \PY{n}{givenFunctions}\PY{p}{,} \PY{n}{givenDerivatives}\PY{p}{,} \PY{n}{delta}\PY{p}{)}
\end{Verbatim}

    For this question, we have to deal with a 2-variable system, so we have
two functions:

    \begin{Verbatim}[commandchars=\\\{\}]
{\color{incolor}In [{\color{incolor}9}]:} \PY{k}{def} \PY{n+nf}{q2Function1}\PY{p}{(}\PY{n}{X}\PY{p}{)}\PY{p}{:} \PY{c+c1}{\PYZsh{} Function for Function 1}
            \PY{n}{x1} \PY{o}{=} \PY{n}{X}\PY{p}{[}\PY{l+m+mi}{0}\PY{p}{]}
            \PY{n}{x2} \PY{o}{=} \PY{n}{X}\PY{p}{[}\PY{l+m+mi}{1}\PY{p}{]}
            \PY{k}{return}\PY{p}{(}\PY{n}{x1}\PY{o}{*}\PY{o}{*}\PY{l+m+mi}{2} \PY{o}{\PYZhy{}} \PY{l+m+mi}{2}\PY{o}{*}\PY{n}{x1} \PY{o}{\PYZhy{}} \PY{n}{x2} \PY{o}{+} \PY{l+m+mf}{0.5}\PY{p}{)}
            
        \PY{k}{def} \PY{n+nf}{q2Derivative11}\PY{p}{(}\PY{n}{X}\PY{p}{)}\PY{p}{:} \PY{c+c1}{\PYZsh{}Partial Derivative of Function 1 on x1}
            \PY{n}{x1} \PY{o}{=} \PY{n}{X}\PY{p}{[}\PY{l+m+mi}{0}\PY{p}{]}
            \PY{n}{x2} \PY{o}{=} \PY{n}{X}\PY{p}{[}\PY{l+m+mi}{1}\PY{p}{]}
            \PY{k}{return}\PY{p}{(}\PY{l+m+mi}{2}\PY{o}{*}\PY{n}{x1} \PY{o}{\PYZhy{}} \PY{l+m+mi}{2}\PY{p}{)}
        
        \PY{k}{def} \PY{n+nf}{q2Derivative12}\PY{p}{(}\PY{n}{X}\PY{p}{)}\PY{p}{:} \PY{c+c1}{\PYZsh{}Partial Derivative of Function 1 on x2}
            \PY{n}{x1} \PY{o}{=} \PY{n}{X}\PY{p}{[}\PY{l+m+mi}{0}\PY{p}{]}
            \PY{n}{x2} \PY{o}{=} \PY{n}{X}\PY{p}{[}\PY{l+m+mi}{1}\PY{p}{]}
            \PY{k}{return}\PY{p}{(}\PY{o}{\PYZhy{}}\PY{l+m+mi}{1}\PY{p}{)}
        
        \PY{k}{def} \PY{n+nf}{q2Function2}\PY{p}{(}\PY{n}{X}\PY{p}{)}\PY{p}{:} \PY{c+c1}{\PYZsh{} Function for Function 2}
            \PY{n}{x1} \PY{o}{=} \PY{n}{X}\PY{p}{[}\PY{l+m+mi}{0}\PY{p}{]}
            \PY{n}{x2} \PY{o}{=} \PY{n}{X}\PY{p}{[}\PY{l+m+mi}{1}\PY{p}{]}
            \PY{k}{return}\PY{p}{(}\PY{n}{x1}\PY{o}{*}\PY{o}{*}\PY{l+m+mi}{2} \PY{o}{+} \PY{l+m+mi}{4}\PY{o}{*}\PY{n}{x2}\PY{o}{*}\PY{o}{*}\PY{l+m+mi}{2} \PY{o}{\PYZhy{}} \PY{l+m+mi}{4}\PY{p}{)}
            
        \PY{k}{def} \PY{n+nf}{q2Derivative21}\PY{p}{(}\PY{n}{X}\PY{p}{)}\PY{p}{:} \PY{c+c1}{\PYZsh{}Partial Derivative of Function 2 on x1}
            \PY{n}{x1} \PY{o}{=} \PY{n}{X}\PY{p}{[}\PY{l+m+mi}{0}\PY{p}{]}
            \PY{n}{x2} \PY{o}{=} \PY{n}{X}\PY{p}{[}\PY{l+m+mi}{1}\PY{p}{]}
            \PY{k}{return}\PY{p}{(}\PY{l+m+mi}{2}\PY{o}{*}\PY{n}{x1}\PY{p}{)}
        
        \PY{k}{def} \PY{n+nf}{q2Derivative22}\PY{p}{(}\PY{n}{X}\PY{p}{)}\PY{p}{:} \PY{c+c1}{\PYZsh{}Partial Derivative of Function 2 on x2}
            \PY{n}{x1} \PY{o}{=} \PY{n}{X}\PY{p}{[}\PY{l+m+mi}{0}\PY{p}{]}
            \PY{n}{x2} \PY{o}{=} \PY{n}{X}\PY{p}{[}\PY{l+m+mi}{1}\PY{p}{]}
            \PY{k}{return}\PY{p}{(}\PY{l+m+mi}{8}\PY{o}{*}\PY{n}{x2}\PY{p}{)}
\end{Verbatim}

    After declaring our functions and partial derivatives, it is now time to
build the functional vector as well as the Jacobian Matrix:

    \begin{Verbatim}[commandchars=\\\{\}]
{\color{incolor}In [{\color{incolor}10}]:} \PY{n}{F} \PY{o}{=} \PY{p}{[}\PY{n}{q2Function1}\PY{p}{,} \PY{n}{q2Function2}\PY{p}{]} \PY{c+c1}{\PYZsh{}Functional vector}
	 \PY{c+c1}{\PYZsh{}Jacobian Matrix}
         \PY{n}{J} \PY{o}{=} \PY{p}{[}\PY{p}{[}\PY{n}{q2Derivative11}\PY{p}{,} \PY{n}{q2Derivative12}\PY{p}{]}\PY{p}{,}\PY{p}{[}\PY{n}{q2Derivative21}\PY{p}{,} \PY{n}{q2Derivative22}\PY{p}{]}\PY{p}{]}
\end{Verbatim}

    \begin{Verbatim}[commandchars=\\\{\}]
{\color{incolor}In [{\color{incolor}11}]:} \PY{n}{x} \PY{o}{=} \PY{n}{np}\PY{o}{.}\PY{n}{linspace}\PY{p}{(}\PY{o}{\PYZhy{}}\PY{l+m+mi}{1}\PY{p}{,}\PY{l+m+mi}{2}\PY{p}{,}\PY{l+m+mi}{2000}\PY{p}{)}
         \PY{n}{y1} \PY{o}{=} \PY{n}{np}\PY{o}{.}\PY{n}{apply\PYZus{}along\PYZus{}axis}\PY{p}{(}\PY{k}{lambda} \PY{n}{x}\PY{p}{:} \PY{n}{x}\PY{o}{*}\PY{o}{*}\PY{l+m+mi}{2} \PY{o}{\PYZhy{}} \PY{l+m+mi}{2}\PY{o}{*}\PY{n}{x} \PY{o}{+} \PY{l+m+mf}{0.5}\PY{p}{,} \PY{l+m+mi}{0}\PY{p}{,} \PY{n}{x}\PY{p}{)}
         \PY{n}{y21} \PY{o}{=} \PY{n}{np}\PY{o}{.}\PY{n}{apply\PYZus{}along\PYZus{}axis}\PY{p}{(}\PY{k}{lambda} \PY{n}{x}\PY{p}{:} \PY{o}{\PYZhy{}}\PY{p}{(}\PY{p}{(}\PY{o}{\PYZhy{}}\PY{n}{x}\PY{o}{*}\PY{o}{*}\PY{l+m+mi}{2} \PY{o}{+} \PY{l+m+mi}{4}\PY{p}{)}\PY{o}{/}\PY{l+m+mi}{4}\PY{p}{)}\PY{o}{*}\PY{o}{*}\PY{l+m+mf}{0.5}\PY{p}{,} \PY{l+m+mi}{0}\PY{p}{,} \PY{n}{x}\PY{p}{)}
         \PY{n}{y22} \PY{o}{=} \PY{n}{np}\PY{o}{.}\PY{n}{apply\PYZus{}along\PYZus{}axis}\PY{p}{(}\PY{k}{lambda} \PY{n}{x}\PY{p}{:} \PY{p}{(}\PY{p}{(}\PY{o}{\PYZhy{}}\PY{n}{x}\PY{o}{*}\PY{o}{*}\PY{l+m+mi}{2} \PY{o}{+} \PY{l+m+mi}{4}\PY{p}{)}\PY{o}{/}\PY{l+m+mi}{4}\PY{p}{)}\PY{o}{*}\PY{o}{*}\PY{l+m+mf}{0.5}\PY{p}{,} \PY{l+m+mi}{0}\PY{p}{,} \PY{n}{x}\PY{p}{)}
         
         \PY{n}{plt}\PY{o}{.}\PY{n}{figure}\PY{p}{(}\PY{l+m+mi}{3}\PY{p}{)}
         \PY{n}{plt}\PY{o}{.}\PY{n}{plot}\PY{p}{(}\PY{n}{x}\PY{p}{,}\PY{n}{y1}\PY{p}{,} \PY{n}{color} \PY{o}{=} \PY{l+s+s1}{\PYZsq{}}\PY{l+s+s1}{b}\PY{l+s+s1}{\PYZsq{}}\PY{p}{,} \PY{n}{label}\PY{o}{=}\PY{l+s+s2}{\PYZdq{}}\PY{l+s+s2}{F1}\PY{l+s+s2}{\PYZdq{}}\PY{p}{)}
         \PY{n}{plt}\PY{o}{.}\PY{n}{plot}\PY{p}{(}\PY{n}{x}\PY{p}{,}\PY{n}{y21}\PY{p}{,} \PY{n}{color} \PY{o}{=} \PY{l+s+s1}{\PYZsq{}}\PY{l+s+s1}{r}\PY{l+s+s1}{\PYZsq{}}\PY{p}{,} \PY{n}{label}\PY{o}{=}\PY{l+s+s2}{\PYZdq{}}\PY{l+s+s2}{F2}\PY{l+s+s2}{\PYZdq{}}\PY{p}{)} \PY{c+c1}{\PYZsh{}Split F2 into 2 to plot the ellipse}
         \PY{n}{plt}\PY{o}{.}\PY{n}{plot}\PY{p}{(}\PY{n}{x}\PY{p}{,}\PY{n}{y22}\PY{p}{,} \PY{n}{color} \PY{o}{=} \PY{l+s+s1}{\PYZsq{}}\PY{l+s+s1}{r}\PY{l+s+s1}{\PYZsq{}}\PY{p}{)} \PY{c+c1}{\PYZsh{}Split F2 into 2 to plot the ellipse}
         \PY{n}{plt}\PY{o}{.}\PY{n}{title}\PY{p}{(}\PY{l+s+s2}{\PYZdq{}}\PY{l+s+s2}{Plot of Question 2 at F(X) = 0}\PY{l+s+s2}{\PYZdq{}}\PY{p}{)}
         \PY{n}{plt}\PY{o}{.}\PY{n}{legend}\PY{p}{(}\PY{p}{)}
         \PY{n}{plt}\PY{o}{.}\PY{n}{xlabel}\PY{p}{(}\PY{l+s+s2}{\PYZdq{}}\PY{l+s+s2}{X}\PY{l+s+s2}{\PYZdq{}}\PY{p}{)}
         \PY{n}{plt}\PY{o}{.}\PY{n}{ylabel}\PY{p}{(}\PY{l+s+s2}{\PYZdq{}}\PY{l+s+s2}{Y}\PY{l+s+s2}{\PYZdq{}}\PY{p}{)}
         \PY{n}{plt}\PY{o}{.}\PY{n}{show}\PY{p}{(}\PY{p}{)}
\end{Verbatim}

    \begin{center}
    \adjustimage{max size={0.9\linewidth}{0.9\paperheight}}{output_23_0.png}
    \end{center}
    { \hspace*{\fill} \\}
    
    By plotting, we now that there are 2 simultaneous roots around
(-0.2,0.9) and (1.9, 0.3).

    Now, calculating approximates roots with threshold deltas:

    \begin{Verbatim}[commandchars=\\\{\}]
{\color{incolor}In [{\color{incolor}12}]:} \PY{n}{cols} \PY{o}{=} \PY{p}{[}\PY{l+s+s2}{\PYZdq{}}\PY{l+s+s2}{Starting Guess}\PY{l+s+s2}{\PYZdq{}}\PY{p}{,} \PY{l+s+s2}{\PYZdq{}}\PY{l+s+s2}{Approximate Roots (x)}\PY{l+s+s2}{\PYZdq{}}\PY{p}{,} \PY{l+s+s2}{\PYZdq{}}\PY{l+s+s2}{Approximate Roots (y)}\PY{l+s+s2}{\PYZdq{}}\PY{p}{,} 
\hspace{4cm} \PY{l+s+s2}{\PYZdq{}}\PY{l+s+s2}{Value of Approx for F1}\PY{l+s+s2}{\PYZdq{}}\PY{p}{,} \PY{l+s+s2}{\PYZdq{}}\PY{l+s+s2}{Value of Approx for F2}\PY{l+s+s2}{\PYZdq{}}\PY{p}{]}
         \PY{n}{data} \PY{o}{=} \PY{p}{[}\PY{p}{]}
         \PY{n}{candidates} \PY{o}{=} \PY{p}{[}\PY{p}{[}\PY{o}{\PYZhy{}}\PY{l+m+mf}{0.2}\PY{p}{,} \PY{l+m+mf}{0.9}\PY{p}{]}\PY{p}{,}\PY{p}{[}\PY{l+m+mf}{1.9}\PY{p}{,} \PY{l+m+mf}{0.3}\PY{p}{]}\PY{p}{]}
         \PY{n}{delta} \PY{o}{=} \PY{l+m+mf}{0.000001}
         
         \PY{k}{for} \PY{n}{c} \PY{o+ow}{in} \PY{n}{candidates}\PY{p}{:}
             \PY{n}{root} \PY{o}{=} \PY{n}{n\PYZus{}Newton}\PY{p}{(}\PY{n}{c}\PY{p}{,} \PY{n}{F}\PY{p}{,} \PY{n}{J}\PY{p}{,} \PY{n}{delta}\PY{p}{)}
             \PY{n}{data}\PY{o}{.}\PY{n}{append}\PY{p}{(}\PY{p}{[}\PY{n}{c}\PY{p}{,}\PY{n}{root}\PY{p}{[}\PY{l+m+mi}{0}\PY{p}{]}\PY{p}{,} \PY{n}{root}\PY{p}{[}\PY{l+m+mi}{1}\PY{p}{]}\PY{p}{,} \PY{n}{q2Function1}\PY{p}{(}\PY{n}{root}\PY{p}{)}\PY{p}{,} \PY{n}{q2Function2}\PY{p}{(}\PY{n}{root}\PY{p}{)}\PY{p}{]}\PY{p}{)}
         
         \PY{n}{q} \PY{o}{=} \PY{n}{pd}\PY{o}{.}\PY{n}{DataFrame}\PY{p}{(}\PY{n}{data}\PY{p}{,} \PY{n}{columns}\PY{o}{=}\PY{n}{cols}\PY{p}{)}
         \PY{n}{q}
\end{Verbatim}

            \begin{Verbatim}[commandchars=\\\{\}]
{\color{outcolor}Out[{\color{outcolor}12}]:}   Starting Guess  Approximate Roots (x)  Approximate Roots (y)  \textbackslash{}
         0    [-0.2, 0.9]              -0.222215               0.993808   
         1     [1.9, 0.3]               1.900677               0.311219   
         
            Value of Approx for F1  Value of Approx for F2  
         0            4.738815e-11            6.167031e-10  
         1            3.866634e-09            5.357242e-08  
\end{Verbatim}
        
    Plotting the approximate roots on the graph:

    \begin{Verbatim}[commandchars=\\\{\}]
{\color{incolor}In [{\color{incolor}13}]:} \PY{n}{x} \PY{o}{=} \PY{n}{np}\PY{o}{.}\PY{n}{linspace}\PY{p}{(}\PY{o}{\PYZhy{}}\PY{l+m+mi}{1}\PY{p}{,}\PY{l+m+mi}{2}\PY{p}{,}\PY{l+m+mi}{2000}\PY{p}{)}
         \PY{n}{y1} \PY{o}{=} \PY{n}{np}\PY{o}{.}\PY{n}{apply\PYZus{}along\PYZus{}axis}\PY{p}{(}\PY{k}{lambda} \PY{n}{x}\PY{p}{:} \PY{n}{x}\PY{o}{*}\PY{o}{*}\PY{l+m+mi}{2} \PY{o}{\PYZhy{}} \PY{l+m+mi}{2}\PY{o}{*}\PY{n}{x} \PY{o}{+} \PY{l+m+mf}{0.5}\PY{p}{,} \PY{l+m+mi}{0}\PY{p}{,} \PY{n}{x}\PY{p}{)}
         \PY{n}{y21} \PY{o}{=} \PY{n}{np}\PY{o}{.}\PY{n}{apply\PYZus{}along\PYZus{}axis}\PY{p}{(}\PY{k}{lambda} \PY{n}{x}\PY{p}{:} \PY{o}{\PYZhy{}}\PY{p}{(}\PY{p}{(}\PY{o}{\PYZhy{}}\PY{n}{x}\PY{o}{*}\PY{o}{*}\PY{l+m+mi}{2} \PY{o}{+} \PY{l+m+mi}{4}\PY{p}{)}\PY{o}{/}\PY{l+m+mi}{4}\PY{p}{)}\PY{o}{*}\PY{o}{*}\PY{l+m+mf}{0.5}\PY{p}{,} \PY{l+m+mi}{0}\PY{p}{,} \PY{n}{x}\PY{p}{)}
         \PY{n}{y22} \PY{o}{=} \PY{n}{np}\PY{o}{.}\PY{n}{apply\PYZus{}along\PYZus{}axis}\PY{p}{(}\PY{k}{lambda} \PY{n}{x}\PY{p}{:} \PY{p}{(}\PY{p}{(}\PY{o}{\PYZhy{}}\PY{n}{x}\PY{o}{*}\PY{o}{*}\PY{l+m+mi}{2} \PY{o}{+} \PY{l+m+mi}{4}\PY{p}{)}\PY{o}{/}\PY{l+m+mi}{4}\PY{p}{)}\PY{o}{*}\PY{o}{*}\PY{l+m+mf}{0.5}\PY{p}{,} \PY{l+m+mi}{0}\PY{p}{,} \PY{n}{x}\PY{p}{)}
         
         \PY{n}{plt}\PY{o}{.}\PY{n}{figure}\PY{p}{(}\PY{l+m+mi}{4}\PY{p}{)}
         \PY{n}{plt}\PY{o}{.}\PY{n}{plot}\PY{p}{(}\PY{n}{x}\PY{p}{,}\PY{n}{y1}\PY{p}{,} \PY{n}{color} \PY{o}{=} \PY{l+s+s1}{\PYZsq{}}\PY{l+s+s1}{b}\PY{l+s+s1}{\PYZsq{}}\PY{p}{,} \PY{n}{label}\PY{o}{=}\PY{l+s+s2}{\PYZdq{}}\PY{l+s+s2}{F1}\PY{l+s+s2}{\PYZdq{}}\PY{p}{)}
         \PY{n}{plt}\PY{o}{.}\PY{n}{plot}\PY{p}{(}\PY{n}{x}\PY{p}{,}\PY{n}{y21}\PY{p}{,} \PY{n}{color} \PY{o}{=} \PY{l+s+s1}{\PYZsq{}}\PY{l+s+s1}{r}\PY{l+s+s1}{\PYZsq{}}\PY{p}{,} \PY{n}{label}\PY{o}{=}\PY{l+s+s2}{\PYZdq{}}\PY{l+s+s2}{F2}\PY{l+s+s2}{\PYZdq{}}\PY{p}{)} \PY{c+c1}{\PYZsh{}Split F2 into 2 to plot the ellipse}
         \PY{n}{plt}\PY{o}{.}\PY{n}{plot}\PY{p}{(}\PY{n}{x}\PY{p}{,}\PY{n}{y22}\PY{p}{,} \PY{n}{color} \PY{o}{=} \PY{l+s+s1}{\PYZsq{}}\PY{l+s+s1}{r}\PY{l+s+s1}{\PYZsq{}}\PY{p}{)} \PY{c+c1}{\PYZsh{}Split F2 into 2 to plot the ellipse}
         \PY{n}{plt}\PY{o}{.}\PY{n}{scatter}\PY{p}{(}\PY{n}{q}\PY{p}{[}\PY{l+s+s2}{\PYZdq{}}\PY{l+s+s2}{Approximate Roots (x)}\PY{l+s+s2}{\PYZdq{}}\PY{p}{]}\PY{p}{,} \PY{n}{q}\PY{p}{[}\PY{l+s+s2}{\PYZdq{}}\PY{l+s+s2}{Approximate Roots (y)}\PY{l+s+s2}{\PYZdq{}}\PY{p}{]}\PY{p}{,}
         \hspace{4cm} \PY{n}{label} \PY{o}{=} \PY{k+kc}{None}\PY{p}{)}
         \PY{n}{plt}\PY{o}{.}\PY{n}{xlabel}\PY{p}{(}\PY{l+s+s2}{\PYZdq{}}\PY{l+s+s2}{X}\PY{l+s+s2}{\PYZdq{}}\PY{p}{)}
         \PY{n}{plt}\PY{o}{.}\PY{n}{ylabel}\PY{p}{(}\PY{l+s+s2}{\PYZdq{}}\PY{l+s+s2}{Y}\PY{l+s+s2}{\PYZdq{}}\PY{p}{)}
         \PY{n}{plt}\PY{o}{.}\PY{n}{title}\PY{p}{(}\PY{l+s+s2}{\PYZdq{}}\PY{l+s+s2}{Plot of Question 2 with Approximate Roots}\PY{l+s+s2}{\PYZdq{}}\PY{p}{)}
         \PY{n}{plt}\PY{o}{.}\PY{n}{legend}\PY{p}{(}\PY{p}{)}
         \PY{n}{plt}\PY{o}{.}\PY{n}{show}\PY{p}{(}\PY{p}{)}
\end{Verbatim}

    \begin{center}
    \adjustimage{max size={0.9\linewidth}{0.9\paperheight}}{output_28_0.png}
    \end{center}

\newpage
    { \hspace*{\fill} \\}
\Large{Summary}
\\ \\
\large{Question 1 Answers}
    \begin{Verbatim}[commandchars=\\\{\}]
{\color{incolor}In [{\color{incolor}14}]:} \PY{n}{p}\PY{p}{[}\PY{p}{[}\PY{l+s+s2}{\PYZdq{}}\PY{l+s+s2}{Approximate Root}\PY{l+s+s2}{\PYZdq{}}\PY{p}{]}\PY{p}{]}
\end{Verbatim}

            \begin{Verbatim}[commandchars=\\\{\}]
{\color{outcolor}Out[{\color{outcolor}14}]:}    Approximate Root
         0         -0.963787
         1         -0.794161
         2         -0.525686
         3         -0.183435
         4          0.183435
         5          0.525686
         6          0.794161
         7          0.963787
\end{Verbatim}
            { \hspace*{\fill} \\}
\large{Question 2 Answers}

    \begin{Verbatim}[commandchars=\\\{\}]
{\color{incolor}In [{\color{incolor}15}]:} \PY{n}{q}\PY{p}{[}\PY{p}{[}\PY{l+s+s2}{\PYZdq{}}\PY{l+s+s2}{Approximate Roots (x)}\PY{l+s+s2}{\PYZdq{}}\PY{p}{,} \PY{l+s+s2}{\PYZdq{}}\PY{l+s+s2}{Approximate Roots (y)}\PY{l+s+s2}{\PYZdq{}}\PY{p}{]}\PY{p}{]}
\end{Verbatim}

            \begin{Verbatim}[commandchars=\\\{\}]
{\color{outcolor}Out[{\color{outcolor}15}]:}    Approximate Roots (x)  Approximate Roots (y)
         0              -0.222215               0.993808
         1               1.900677               0.311219
\end{Verbatim}
        

    % Add a bibliography block to the postdoc
    
    
    
    \end{document}
